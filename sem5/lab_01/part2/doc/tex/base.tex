\chapter{Функции обработчика прерывания от системного таймера в защищенном режиме}

\section{ОС семейства Unix}

\subsection{По тику}

Период времени между двумя последующими прерываниями таймера называется \textit{тиком}. В задачи обработчика прерывания от системного таймера по тику входит:

\begin{itemize}
	\item инкремент счетчика тиков аппаратного таймера;
	\item инкремент часов и других таймеров системы;
	\item декремент кванта;
	\item декремент счетчика времени до отправления на выполнение отложенного вызова;
	\item обновление статистики использования процессора текущим процессом : инкремент поля p\_cpu структуры proc.
\end{itemize}

\subsection{По главному тику}

Период времени, который равен \textit{n} тикам таймера (число \textit{n} зависит от конкретного варианта системы), называют \textit{главным тиком}. В задачи обработчика прерывания от системного таймера по главному тику входит:

\begin{itemize}
	\item регистрация отложенных вызовов функций, относящихся к работе планировщика;
	\item пробуждение системных процессов - распознавание отложенного вызова процедуры \textit{wakeup}, которая перемещает дескрипторы процессов из списка ''спящих'' в очередь ''готовых к выполнению'';
	\item декремент счетчика времени, которое осталось до отправления одного из следующих сигналов:
	\begin{itemize}
		\item \textit{SIGALRM} - сигнал, посылемый процессу по истечении времени, которое предварительно задано функцией \textit{alarm()};
		\item \textit{SIGPROF} - сигнал, посылаемый процессу по истечении времени, которое задано в таймере профилирования;
		\item \textit{SIGVTALRM} - сигнал, посылаемый процессу по истечении времени, которое задано в ''виртуальном'' таймере.
	\end{itemize}
\end{itemize}

\subsection{По кванту}

Временной интервал, в течение которого процесс может использовать процессор до вытеснения другим процессом, называют \textit{квантом времени}. Обработчик прерывания от системного таймера по кванту посылает сигнал \textit{SIGXCPU} текущему процессу, если он израсходовал выделенный ему квант времени.

\section{ОС семейства Windows}

\subsection{По тику}

В задачи обработчика прерывания от системного таймера по тику входит:

\begin{itemize}
	\item инкремент счетчика системного времени;
	\item декремент кванта текущего потока на величину, равную количеству
          тактов процессора, произошедших за тик;
	\item декремент счетчика времени отложенных задач;
	\item инициализация отложенного вызова обработчика ловушки профилирования ядра путем постановки объекта в очередь DPC, если активен механизм профилирования ядра.
\end{itemize}

\subsection{По главному тику}

Обработчик прерывания от системного таймера по кванту освобождает объект "событие"{}, который ожидает диспетчер настройки баланса.


\subsection{По кванту}

Обработчик прерывания от системного таймера по кванту инициализирует диспетчеризацию потоков путем постановки соответствующего объекта в очередь DPC.

\chapter{Пересчет динамических приоритетов}

В ОС семейства Unix и в ОС семейства Unix динамически пересчитываться могут только приоритеты пользовательских процессов.

\section{ОС семейства Unix}

Очередь процессов, готовых к выполнению, формируется согласно приоритетам процессов и принципу вытесняющего циклического планирования.

Приоритет задается целым числом из диапазона от 0 до 127. Чем меньше число, тем выше приоритет процесса. Приоритеты ядра находятся в диапазоне от 0 до 49 и являются фиксированными величинами, а приоритеты прикладных задач - в диапазоне от 50 до 127.

Процессы с большим приоритетом выполняются в первую очередь, процессы с одинаковыми приоритетами выполняются в течении кванта времени циклически друг за другом. Когда в очередь поступает процесс с более высоким приоритетом, планировщик предоставляет ресурс более приоритетному процессу, вытесняя текущий.

Традиционное ядро Unix - строго невытесняемое, то есть, ядро не заставит процесс, выполняющийся в режиме ядра, уступить процессор какому-либо высокоприоритетному процессу. В современных системах Unix ядро - вытесняющее: процесс в режиме ядра может быть вытеснен более приоритетным процессом в режиме ядра. Ядро было сделано вытесняющим, чтобы система могла обслуживать процессы реального времени (видео или аудио).

Приоритеты прикладных задач могут изменяться во времени в зависимости от двух факторов:

\begin{itemize}
	\item фактор любезности. Чем меньше его значение, тем выше приоритет процесса. Суперпользователь может повысить приоритет с помощью системного вызова \textit{nice};
	\item последней измеренной величины использования процессора.
\end{itemize}

Структура \textit{proc} содержит следующие поля, относящиеся к приоритету:
\begin{itemize}
    \item \textit{\_pri} – текущий приоритет планирования;
    \item \textit{p\_usrpri} – приоритет процесса в режиме задачи;
    \item \textit{p\_cpu} – результат последнего измерения степени загруженности процессора процессом;
    \item \textit{p\_nice} – фактор любезности.
\end{itemize}

У процесса, который находится в режиме задачи, значения \textit{p\_pri} и \textit{p\_usrpri} равны. Значение текущего приоритета \textit{p\_pri} может быть повышено планировщиком для выполнения процесса в режиме ядра (\textit{p\_usrpri} будет использоваться для хранения приоритета, который будет назначен при возврате в режим задачи).

Ядро системы связывает приоритет сна с событием или ожидаемым ресурсом, из-за которого процесс может блокироваться. Когда процесс ''просыпается'' после
блокирования в системном вызове, ядро устанавливает в поле \textit{p\_pri} приоритет сна – значение приоритета из диапазона от 0 до 49, зависящее от события или ресурса по которому произошла блокировка. Значения приоритетов сна для событий в системе \textit{4.3BSD} описаны в таблице \ref{tbl:bsd}.

\begin{table}[h]
    \begin{center}
        \begin{threeparttable}
        \captionsetup{justification=raggedright,singlelinecheck=off}
        \caption{Приоритеты сна в 4.3BSD}
        \label{tbl:bsd}
        \begin{tabular}{|c|c|c|}
            \hline
            Приоритет & Значение & Описание \\
            \hline
            PSWP & 0 & Свопинг \\ \hline  
            PSWP + 1 & 1 & Страничный демон \\ \hline
            PSWP + 1/2/4 & 1/2/4 & Другие действия при обработке памяти \\ \hline
            PINOD & 10 & Ожидание освобождения inode  \\ \hline 
            PRIBIO & 20 & Ожидание дискового ввода-вывода \\ \hline
            PRIBIO + 1 & 21 & Ожидание освобождения буфера \\ \hline
            PZERO & 25 & Базовый приоритет \\ \hline
            TTIPRI & 28 & Ожидание ввода с терминала \\ \hline
            TTORPI & 29 & Ожидание вывода с терминала \\ \hline                                             
		\end{tabular}
    \end{threeparttable}
\end{center}
\end{table}

Системные приоритеты сна для \textit{4.3BSD UNIX} и \textit{SCO UNIX} приведены в таблице \ref{tbl:book}.

\begin{table}[h]
    \begin{center}
        \begin{threeparttable}
        \captionsetup{justification=raggedright,singlelinecheck=off}
        \caption{Системные приоритеты сна}
        \label{tbl:book}
        \begin{tabular}{|p{80mm}|p{40mm}|p{30mm}|}
            \hline
            Событие & Приоритет 4.3BSD UNIX  & Приоритет SCO UNIX \\ \hline
            Ожидание загрузки в память сегмента/страницы (свопинг/страничное замещение) & 0 & 95\\ \hline
            Ожидание индексного дескриптора & 10 & 88\\ \hline
            Ожидание ввода/вывода & 20 & 81\\ \hline
            Ожидание буфера & 30 & 80\\ \hline
            Ожидание терминального ввода & & 75\\ \hline
            Ожидание терминального вывода &  & 74\\ \hline
            Ожидание завершения выполнения & & 73\\ \hline
            Ожидание события -- низкоприоритетное состояние сна & 40 & 66 \\ \hline                                            
		\end{tabular}
    \end{threeparttable}
\end{center}
\end{table}

При создании процесса поле \textit{p\_cpu} инициализируется нулем. На каждом тике обработчик таймера увеличивает поле \textit{p\_cpu} текущего процесса на единицу, до максимального значения, равного 127. Каждую секунду, обработчик прерывания инициализирует отложенный вызов процедуры \textit{schedcpu()}, которая уменьшает значение \textit{p\_cpu} каждого процесса исходя из фактора ''полураспада''.

В системе \textit{4.3BSD} для расчёта фактора полураспада применяется формула
\eqref{for:bsd}.

\begin{equation}
    \label{for:bsd}
    decay = \frac{2 \cdot load\_average}{2 \cdot load\_average + 1}
\end{equation}

где \textit{load\_average} - это среднее количество процессов, находящихся в состоянии готовности к выполнению, за последнюю секунду.

Процедура \textit{schedcpu()} пересчитывает приоритеты для режима задачи всех процессов по формуле \eqref{for:task}.

\begin{equation}
    \label{for:task}
    p\_usrpri = PUSER + \frac{p\_cpu}{2} + 2 \cdot p\_nice
\end{equation}

где \textit{PUSER} - базовый приоритет в режиме задачи, равный 50.

\textit{p\_cpu} будет увеличен, если процесс в последний раз использовал большое количество процессорного времени. Это приведет к росту значения \textit{p\_usrpri}, из чего следует понижение приоритета. Чем дольше процесс простаивает в очереди на исполнение, тем больше фактор
полураспада уменьшает его \textit{p\_cpu}, что приводит к повышению его приоритета. Такая схема предотвращает бесконечное откладывание низкоприоритетных процессов в ОС. Ее применение предпочтительнее процессам, осуществляющим много операций ввода-вывода, в противоположность процессам, производящим много вычислений.


\section{ОС семейства Windows}

В Windows процессу при создании назначается базовый приоритет. Относительно базового приоритета процесса потоку назначается относительный приоритет. 

Планирование осуществляется только на основании приоритетов потоков, готовых к выполнению: если поток с более высоким приоритетом становится готовым к выполнению, поток с более низким приоритетом вытесняется планировщиком. По истечению кванта времени текущего потока, ресурс передается самому приоритетному потоку в очереди готовых к выполнению.

В Windows используется 32 уровня приоритета - целое число от 0 до 31 (наивысший):
\begin{itemize}
    \item от 0 до 15 - изменяющиеся уровни. Уровень 0 зарезервирован для потока обнуления страниц;
    \item от 16 до 31 - уровни реального времени.
\end{itemize}

Уровни приоритета потоков назначаются в зависимости от двух разных позиций: Windows API и ядра Windows.

Сначала Windows API систематизирует процессы по классу приоритета, который им присваивается при создании:
\begin{itemize}
    \item реального времени (Real-time) - (4);
    \item высокий (High) - (3);
    \item выше обычного (Above Normal) - (6);
    \item обычный (Normal) - (2);
    \item ниже обычного (Below Normal) - (5);
    \item уровень простоя (Idle) - (1).
\end{itemize}

Затем назначается относительный приоритет отдельных потоков внутри этих процессов:
\begin{itemize}
    \item критичный по времени (Time-critical) - (15);
    \item наивысший (Highest) - (2);
    \item выше обычного (Above-normal) - (1);
    \item обычный (Normal) - (0);
    \item ниже обычного (Below-normal) - (–1);
    \item самый низший (Lowest) - (–2);
    \item уровень простоя (Idle) - (–15)
\end{itemize}

Процесс по умолчанию наследует свой базовый приоритет у того процесса, который его создал.

В таблице \ref{tab:prioritet} показано соответствие между приоритетами Windows API и ядра системы.

\begin{table}[h]
    \caption{Соответствие между приоритетами Windows API и ядра Windows}
    \begin{center}
        \begin{tabular}{|c|c|c|c|c|c|c|}
            \hline
            Класс приоритета & Real-time & High & Above &
            Normal & Below Normal & Idle \\ \hline
            Time Critical & 31 & 15 & 15 & 15 & 15 & 15 \\ \hline
            Highest & 26 & 15 & 12 & 10 & 8 & 6 \\ \hline
            Above Normal & 25 & 14 & 11 & 9 & 7 & 5 \\ \hline
            Normal & 24 & 13 & 10 & 8 & 6 & 4 \\ \hline
            Below Normal & 23 & 12 & 9 & 7 & 5 & 3 \\ \hline
            Lowest & 22 & 11 & 8 & 6 & 4 & 2 \\ \hline
            Idle & 16 & 1 & 1 & 1 & 1 & 1 \\ \hline
        \end{tabular}
    \end{center}
    \label{tab:prioritet}
\end{table}

В Windows также включен \textit{диспетчер настройки баланса}. Он раз в секунду сканирует очередь готовых потоков. Если обнаружены потоки, ожидающие выполнения более 4 секунд, диспетчер настройки баланса повышает их приоритет до 15. Когда истекает квант, приоритет потока снижается до базового приоритета. Если поток не был завершен за квант времени или был вытеснен потоком с более высоким приоритетом, то после снижения приоритета поток возвращается в очередь готовых потоков.

Для того, чтобы минимизировать расход процессорного времени, диспетчер настройки баланса сканирует только 16 потоков и повышает приоритет не более чем у 10 потоков за одни проход. Когда обнаружится 10 потоков, приоритет которых следует повысить, диспетчер настройки баланса прекращает сканирование. При следующем проходе возобновляет сканирование с того места, где оно было прервано.

Планировщик может повысить текущий приоритет потока в динамическом диапазоне (от 1 до 15) вследствие следующих причин:

\begin{itemize}
    \item повышение вследствие событие планировщика или диспетчера;
    \item повышение приоритета владельца блокировки;
    \item повышение приоритета после завершения ввода/вывода (таблица \ref{tab:input-output});
    
\begin{table}[h]
    \caption{Рекомендуемые значения повышения приоритета}
    \begin{center}
        \begin{tabular}{|p{100mm}|l|}
            \hline
            \textbf{Устройство} & \textbf{Приращение} \\\hline
            Диск, CD-ROM, параллельный порт, видео & 1 \\ \hline
            Сеть, почтовый ящик, именованный канал, последовательный порт & 2 \\ \hline
            Клавиатура, мышь & 6 \\ \hline
            Звуковая плата & 8 \\ \hline
        \end{tabular}
    \end{center}
    \label{tab:input-output}
\end{table}
    \item повышение приоритета вследствие ввода из пользовательского интерфейса;
    \item повышение приоритета вследствие длительного ожидания ресурса исполняющей системы;
    \item повышение вследствие ожидания объекта ядра;
    \item повышение приоритета в случае, когда готовый к выполнению поток не был запущен в течение длительного времени;
    \item повышение приоритета проигрывания для мультимедийных приложений. Для того, чтобы мультимедийные потоки могли выполняться с минимальными задержками, функции \textit{MMCSS} (\textit{MultiMedia Class Scheduler Service}) временно повышают приоритет потоков до уровня, соответствующего их категориям планирования (таблица \ref{tab:category}). Для того, чтобы другие потоки могли получить ресурс, приоритет снижается до уровня, соответствующего категории \textit{Exhausted}.
\end{itemize}

\begin{table}[h]
    \caption{Категории планирования}
    \begin{center}
        \begin{tabular}{|p{40mm}|p{30mm}|p{80mm}|}
            \hline
            \textbf{Категория} & \textbf{Приоритет} & \textbf{Описание} \\
            \hline
            High (Высокая) & 23-26 & Потоки профессионального аудио (Pro
Audio), запущенные с приоритетом выше, чем у других потоков на системе, за
исключением критических системных потоков \\
            \hline
            Medium (Средняя) & 16-22 & Потоки, являющиеся частью приложений
первого плана, например Windows Media Player \\
            \hline
            Low (Низкая) & 8-15 & Все остальные потоки, не являющиеся частью
предыдущих категорий \\
            \hline
            Exhausted (Исчерпавших потоков) & 1-7 & Потоки, исчерпавшие свою
долю времени центрального процессора, выполнение которых продолжиться, только
если не будут готовы к выполнению другие потоки с более высоким уровнем
приоритета \\
            \hline
        \end{tabular}
    \end{center}
    \label{tab:category}
\end{table}