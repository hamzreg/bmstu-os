\chapter{Аналитическая часть}

В данном разделе будут описаны последовательный и конвейерный варианты алгоритма. Также будут изучены алгоритмы, выполняемые на отдельных лентах конвейера.

\section{Последовательный алгоритм}

В некоторых задачах происходит обработка текстовых данных перед их использованием в решении. Данные проходят ряд преобразований в несколько последовательных этапов. Каждый этап реализуется при помощи алгоритма работы со строками.

Строка \cite{str} — это последовательность ASCII или UNICODE символов. В некоторых системных языках программирования в качестве строк используются массивы символов. В этом случае с символами строк можно работать как с элементами массива.

В данной лабораторной работе будет реализована обработка строк или массивов символов, состоящая из следующих этапов:
\begin{itemize}
	\item удаление пробелов в строке;
	\item перевод каждого символа строки в другой регистр;
	\item определение строки-палиндрома.
\end{itemize}

\section{Алгоритмы, выполняемые на отдельных лентах}

Каждый из описанных выше этапов будет выполняться на отдельной ленте.

\subsection{Удаление пробелов в строке}

Удаление пробелов в строке будет происходить следующим образом:

\begin{itemize}
	\item каждый символ строки сравнивается с символом пробела;
	\item символ удаляется из строки при совпадении с пробелом.
\end{itemize}

\subsection{Перевод символов строки в другой регистр}

В данном алгоритме выполняются следующие действия для каждого символа:

\begin{itemize}
	\item если символ в верхнем регистре, то он заменяется на символ в нижнем регистре;
	\item если символ в нижнем регистре, то он заменяется на символ в верхнем регистре.
\end{itemize}

\subsection{Определение строки-палиндрома}

Палиндром - это симметричный относительно своей середины набор символов. Чтобы определить, является ли строка $s$ длины $len$ палиндромом необходимо:

\begin{itemize}
	\item каждый символ строки $s[i]$ сравнить с символом $s[len - i - 1]$,  для $i \in [0;len/2]$;
	\item если текущая пара символов не совпадает, то строка не является палиндромом;
	\item если все символы из заданного промежутка $i$ совпали, то строка - палиндром.
\end{itemize}

Таким образом, в последовательном алгоритме:
\begin{itemize}
	\item строки из набора алгоритма добавляются в первую очередь;
	\item из первой очереди извлекается строка $s1$ и обрабатывается на первой ленте;
	\item строка $s1$ добавляется во вторую очередь;
	\item из второй очереди извлекается строка $s1$ и обрабатывается на второй ленте;
	\item строка $s1$ добавляется в третью очередь;
	\item из третьей очереди извлекается строка $s1$ и обрабатывается на третьей ленте;
	\item из первой очереди извлекается строка $s2$ и обрабатывается на первой ленте;
	\item далее каждая строка из набора последовательно повторяет цикл обработки.
\end{itemize}

\section{Параллельный алгоритм}

При помощи многопоточности можно ускорить процесс обработки строк. В этом случае для каждой ленты создается отдельный поток. Извлечение и добавление строк осуществляется потоками. Так, в параллельном алгоритме:

\begin{itemize}
	\item первый поток извлекает строку из первой очереди, она обрабатывается на первой ленте, и поток добавляет ее во вторую очередь;
	\item второй поток извлекает строку из второй очереди, она обрабатывается на второй ленте, и поток добавляет ее в третью очередь;
	\item третий поток извлекает строку из второй очереди, и она обрабатывается на третьей ленте;
	\item каждый поток по завершении вновь извлекает строку из своей очереди, и цикл обработки повторяется в параллельном режиме.
\end{itemize}

\section{Вывод}

Для реализации последовательной и конвейерной обработки данных был выбран алгоритм работы со строкой. На трех лентах выполняются следующие этапы алгоритма:

\begin{itemize}
	\item удаление пробелов в строке;
	\item перевод каждого символа строки в другой регистр;
	\item определение строки-палиндрома.
\end{itemize}

Программе, реализующей данный алгоритм, на вход будет подаваться набор строк определенной длины. Программа должна работать в рамках следующих ограничений:

\begin{itemize}
	\item символами строк могут быть буквы латинского алфавита в верхнем и нижнем регистре и пробел;
	\item должно быть выдано сообщение об ошибке при вводе не числовых или неположительных длины и числа строк.
\end{itemize}

Пользователь должен иметь возможность выбора обработки строк: последовательно или при помощи конвейера. Также должны быть реализованы сравнение вариантов обработки по времени работы в зависимости от количества строк в наборе и в зависимости от длины строк с выводом результатов на экран и получение графического представления результатов сравнения. Результат данных действий пользователь должен получать при помощи меню.