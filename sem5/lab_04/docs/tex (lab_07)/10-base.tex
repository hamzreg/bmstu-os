\section*{Первая программа}

В первой программе выполняется: чтение массива из файла, сортировка массива по возрастанию и вывод массива на экран.

Данная программа выполнялась на лабораторной работе в курсе "Программирование на языке C".

\begin{center}
    \captionsetup{justification=raggedright,singlelinecheck=off}
    \begin{lstlisting}[label=lst:array,caption=array.c]
#include <stdio.h>
#include <stdlib.h>
#include "../inc/array.h"


int get_array_len(FILE *in_file, array_t *const array)
{
    if (array == NULL)
    {
        LOG_ERROR("%s", "Invalid pointer");
        return ERROR_POINTER;
    }

    int array_len = 0, number;

    if (fscanf(in_file, "%d", &number) == 1)
    {
        array_len++;

        while (fscanf(in_file, "%d", &number) == 1)
            array_len++;

        array->len = array_len;

        return OK;
    }

    LOG_ERROR("%s", "Data read error");
    return DATA_ERROR;
}


int fill_array(FILE *in_file, int *const first, int *const last)
{
    if (first == NULL || last == NULL)
    {
        LOG_ERROR("%s", "Invalid pointer");
        return ERROR_POINTER;
    }

    if (first == last)
    {
        LOG_ERROR("%s", "Empty array");
        return EMPTY_ARRAY;
    }

    int number;

    for (int *array_pointer = first; array_pointer != last; array_pointer++)
    {
        if (fscanf(in_file, "%d", &number))
            *array_pointer = number;
    }

    return OK;
}


void even_passage(char *start, char *end, size_t size, int (*compar)(const void *, const void *))
{
    for (char *array_pointer = end; array_pointer > start; array_pointer -= size)
        if (compar(array_pointer, array_pointer - size) <= 0)
            swap(size, array_pointer, array_pointer - size);
}


void odd_passage(char *start, char *end, size_t size, int (*compar)(const void *, const void *))
{
    for (char *array_pointer = start; array_pointer < end; array_pointer += size)
        if (compar(array_pointer, array_pointer + size) >= 0)
            swap(size, array_pointer, array_pointer + size);
}


void mysort(void *base, size_t nmemb, size_t size, int (*compar)(const void *, const void *))
{
    if (base == NULL || nmemb <= 0)
        return;

    char *start = base;
    char *end = start + size * (nmemb - 1);

    for (size_t i = 0; i < nmemb; i++)
    {
        if (!(i % 2))
            even_passage(start, end, size, compar);
        else
            odd_passage(start, end, size, compar);
    }
}


int key(const int *pb_src, const int *pe_src, int **pb_dst, int **pe_dst)
{
    if (pb_src == NULL || pe_src == NULL)
    {
        LOG_ERROR("%s", "Invalid pointer");
        return ERROR_POINTER;
    }

    if (pb_src == pe_src)
    {
        LOG_ERROR("%s", "Empty array");
        return EMPTY_ARRAY;
    }


    int filter_len = 0;

    for (const int *array_pointer = pb_src; array_pointer < pe_src - 1; array_pointer++)
    {
        if (sum_numbers(array_pointer + 1, pe_src) < *array_pointer)
            filter_len++;
    }

    if (!filter_len)
    {
        LOG_ERROR("%s", "There are no such elements");
        return ERROR_FILTER;
    }
    LOG_INFO("Count such elements is %d", filter_len);

    *pb_dst = malloc(filter_len * sizeof(int));

    if (pb_dst == NULL)
    {
        LOG_ERROR("%s", "Memory allocation error");
        return MEMORY_ERROR;
    }
    LOG_INFO("%s", "New array created");

    *pe_dst = *pb_dst + filter_len;

    int *new_array_pointer = *pb_dst;
    for (const int *array_pointer = pb_src; array_pointer < pe_src - 1; array_pointer++)
    {
        if (sum_numbers(array_pointer + 1, pe_src) < *array_pointer)
        {
            *new_array_pointer = *array_pointer;
            new_array_pointer++;
        }
    }
    LOG_INFO("%s", "Numbers written");

    return OK;
}


void write_array_file(FILE *out_file, const int *const first, const int *const second)
{
    for (const int *pa = first; pa != second; pa++)
        fprintf(out_file, "%d ", *pa);
}

void print_array(const int *const first, const int *const second)
{
    for (const int *pa = first; pa != second; pa++)
        printf("%d ", *pa);
    
    printf("\n");
}
\end{lstlisting}
\end{center}

\begin{center}
    \captionsetup{justification=raggedright,singlelinecheck=off}
    \begin{lstlisting}[label=lst:numbers,caption=numbers.c]
#include <stdio.h>
#include "../inc/numbers.h"


int compare_int(const void *first, const void *second)
{
    if (first == NULL || second == NULL)
    {
        LOG_ERROR("%s", "Invalid pointer");
        return ERROR_POINTER;
    }

    return *(int *)first - *(int *)second;
}


void swap(size_t size, char *number_one, char *number_two)
{
    char add_number;

    while (size > 0)
    {
        add_number = *number_one;
        *number_one = *number_two;
        *number_two = add_number;

        number_one++;
        number_two++;
        size--;
    }
}


int sum_numbers(const int *first, const int *last)
{
    if (first == NULL || last == NULL)
    {
        LOG_ERROR("%s", "Invalid pointer");
        return ERROR_POINTER;
    }

    if (first == last)
    {
        LOG_ERROR("%s", "Empty array");
        return EMPTY_ARRAY;
    }

    int sum = 0;

    for (const int *array_pointer = first; array_pointer < last; array_pointer++)
        sum += *array_pointer;

    return sum;
}
\end{lstlisting}
\end{center}

\clearpage

\begin{center}
    \captionsetup{justification=raggedright,singlelinecheck=off}
    \begin{lstlisting}[label=lst:main,caption=main.c]
#include <stdio.h>
#include <stdlib.h>
#include <string.h>
#include "../inc/array_t.h"
#include "../inc/array.h"
#include "../inc/numbers.h"
#include "../inc/errors.h"
#include "../inc/macrologger.h"

int main(int argc, char **argv)
{
    if (argc < 2 || argc > 3)
    {
        LOG_ERROR("%s", "Invalid number of arguments");
        return ERROR_ARGC;
    }


    FILE *in_file = fopen(argv[1], "r");

    if (in_file == NULL)
    {
        LOG_ERROR("%s", "File open error");
        return FILE_OPEN_ERROR;
    }
    LOG_INFO("%s", "Input file open");


    array_t array;
    int code = get_array_len(in_file, &array);
    
    if (code != OK)
        return code;
    LOG_INFO("Array len is %d", array.len);

    array.start = malloc(array.len * sizeof(int));
         
    if (array.start == NULL)
    {
        LOG_ERROR("%s", "Memory allocation error");
        return MEMORY_ERROR;
    }
    LOG_INFO("%s", "Array created");

    rewind(in_file);
    code = fill_array(in_file, array.start, array.start + array.len);

    if (code != OK)
    {
        free(array.start);
        return code;
    }
    LOG_INFO("%s", "Array filled");


    if (fclose(in_file))
    {
        free(array.start);

        LOG_ERROR("%s", "File close error");
        return FILE_CLOSE_ERROR;
    }
    LOG_INFO("%s", "Input file close");


    if (argc == 2)
    {
        printf("Before sort: ");
        print_array(array.start, array.start + array.len);
        mysort(array.start, array.len, sizeof(int), compare_int);
        LOG_INFO("%s", "Array is sorted");

        printf("After sort: ");
        print_array(array.start, array.start + array.len);
        free(array.start);
    }


    if (argc == 3)
    {
        if (strcmp("f", argv[2]) != OK)
        {
            free(array.start);
            LOG_ERROR("%s", "Invalid command");
            return ERROR_COMMAND;
        }


        array_t new_array;
        int *end;

        code = key(array.start, array.start + array.len, &new_array.start, &end);
        free(array.start);

        if (code != OK)
            return code;
        LOG_INFO("%s", "Array filtered");

        new_array.len = end - new_array.start;
        LOG_INFO("New array len is %d", new_array.len);

        mysort(new_array.start, new_array.len, sizeof(int), compare_int);
        LOG_INFO("%s", "Array is sorted");
        
        print_array(new_array.start, new_array.start + new_array.len);
        free(new_array.start);
    }
    LOG_INFO("%s", "Numbers written to file");
   
    return OK;
}
\end{lstlisting}
\end{center}

\clearpage

\begin{center}
    \captionsetup{justification=raggedright,singlelinecheck=off}
    \begin{lstlisting}[label=lst:arrayt,caption=array\_t.h]
#ifndef ARRAY_T_H
#define ARRAY_T_H

typedef struct
{
    int *start;
    int len;
} array_t;

#endif
\end{lstlisting}
\end{center}

\begin{center}
    \captionsetup{justification=raggedright,singlelinecheck=off}
    \begin{lstlisting}[label=lst:array,caption=array.h]
#ifndef ARRAY_H
#define ARRAY_H

#include "array_t.h"
#include "numbers.h"
#include "errors.h"
#include "macrologger.h"


int get_array_len(FILE *in_file, array_t *const array);

int fill_array(FILE *in_file, int *const first, int *const last);

void mysort(void *base, size_t nmemb, size_t size, int (*compar)(const void *, const void *));

int key(const int *pb_src, const int *pe_src, int **pb_dst, int **pe_dst);

void write_array_file(FILE *out_file, const int *const first, const int *const second);

void print_array(const int *const first, const int *const second);

#endif
\end{lstlisting}
\end{center}

\begin{center}
    \captionsetup{justification=raggedright,singlelinecheck=off}
    \begin{lstlisting}[label=lst:numbers,caption=numbers.h]
#ifndef NUMBERS_H
#define NUMBERS_H

#include "errors.h"
#include "macrologger.h"

int compare_int(const void *first, const void *second);

void swap(size_t size, char *number_one, char *number_two);

int sum_numbers(const int *first, const int *last);

#endif

\end{lstlisting}
\end{center}

\begin{center}
    \captionsetup{justification=raggedright,singlelinecheck=off}
    \begin{lstlisting}[label=lst:errors,caption=errors.h]
#ifndef ERRORS_H
#define ERRORS_H

#define ERROR_ARGC 1
#define FILE_OPEN_ERROR 2
#define MEMORY_ERROR 3
#define FILE_CLOSE_ERROR 4
#define DATA_ERROR 5
#define ERROR_FILTER 6
#define ERROR_COMMAND 7
#define ERROR_POINTER 8
#define EMPTY_ARRAY 9

#define OK 0

#endif
\end{lstlisting}
\end{center}