\section*{Вторая программа}

Во второй программе выполняется: чтение последовательности из файла и вывод на экран информации о том, является ли последовательность палиндромом. Для работы с последовательностью используется список.

Данная программа выполнялась на лабораторной работе в курсе "Программирование на языке C".

\begin{center}
    \captionsetup{justification=raggedright,singlelinecheck=off}
    \begin{lstlisting}[label=lst:list,caption=list.c]
#include "../inc/list.h"

node_t *reverse(node_t *head)
{
    if (!head)
        return NULL;

    node_t *result = NULL;
    node_t *node;

    while (head != NULL)
    {
        void *element = pop_front(&head);
        node = create_node(element);
        add_node(&result, node);
    }

    return result;
}


void sorted_insert(node_t **head, node_t *element,
int (*comparator)(const void *, const void *))
{
    element->next = NULL;

    if (!(*head))
    {
        element->next = *head;
        *head = element;

        return;
    }

    char *first, *second = element->data;

    if ((*head)->next == NULL)
    {
        first = (*head)->data;

        if (comparator(first, second) <= 0)
        {
            (*head)->next = element;
            element->next = NULL;

            return;
        }

        element->next = *head;
        *head = element;

        return;
    }

    node_t *node = *head, *add;
    first = node->data;

    if (comparator(first, second) > 0)
    {
        add = *head;
        *head = element;
        element->next = add;

        return;
    }

    while (node->next)
    {
        first = node->next->data;

        if (comparator(first, second) > 0)
        {
            add = node->next;
            node->next = element;
            element->next = add;

            return;
        }

        node = node->next;
    }

    node->next = element;
}


node_t *sort(node_t *head, int (*comparator)(const void *, const void *))
{
    if (!head)
        return NULL;

    if (!comparator)
        return NULL;

    node_t *result = NULL;

    while (head != NULL)
    {
        node_t *node = head;
        head = head->next;
        sorted_insert(&result, node, comparator);
    }

    return result;
}


int compare_int(const void *first, const void *second)
{
    return *(int *)first - *(int *)second;
}
\end{lstlisting}
\end{center}

\begin{center}
    \captionsetup{justification=raggedright,singlelinecheck=off}
    \begin{lstlisting}[label=lst:node,caption=node.c]
#include "../inc/node.h"

node_t *create_node(void *data)
{
    node_t *node = malloc(sizeof(node_t));

    if (node)
    {
        node->data = data;
        node->next = NULL;
    }

    return node;
}


void add_node(node_t **head, node_t *node)
{
    node->next = *head;
    *head = node;
}


void *pop_front(node_t **head)
{
    if (!head || !(*head))
        return NULL;

    node_t *node = (*head);
    void *data = node->data;
    (*head) = (*head)->next;

    free(node);

    return data;
}


void *pop_back(node_t **head)
{
    if (!head || !(*head))
        return NULL;

    node_t *node = *head, *prev_node = NULL;

    for (; node->next; node = node->next)
        prev_node = node;

    void *data = node->data;

    if (prev_node)
        prev_node->next = node->next;
    else
        *head = node->next;

    free(node);

    return data;
}
\end{lstlisting}
\end{center}


\begin{center}
    \captionsetup{justification=raggedright,singlelinecheck=off}
    \begin{lstlisting}[label=lst:sequence,caption=sequence.c]
#include "../inc/sequence.h"

void init_sequence(sequence_t *sequence)
{
    sequence->count = 0;
    sequence->head = NULL;
}


int get_count_elements(FILE *in_file, int *count)
{
    int element;

    while (fscanf(in_file, "%d", &element) == 1)
        (*count)++;

    if (!*count)
        return EMPTY_FILE;

    return OK;
}


int read_sequence(const char *filename, sequence_t *sequence,
int **storage)
{
    FILE *in_file = fopen(filename, "r");

    if (!in_file)
    {
        LOG_ERROR("%s", "File open error");
        return ERROR_FILE_OPEN;
    }

    init_sequence(sequence);

    if (get_count_elements(in_file, &sequence->count))
        return EMPTY_FILE;

    rewind(in_file);

    *storage = malloc(sequence->count * sizeof(int));

    if (!*storage)
        return ALLOCATE_ERROR;

    int element;
    int i = 0;

    while (fscanf(in_file, "%d", &element) == 1)
    {
        (*storage)[i] = element;
        i++;
    }

    create_sequence(sequence, *storage);

    if (fclose(in_file) != OK)
    {
        free_sequence(sequence->head);
        free(*storage);

        LOG_ERROR("%s", "File close error");
        return ERROR_FILE_CLOSE;
    }

    return OK;
}


void create_sequence(sequence_t *sequence, int *storage)
{
    for (int i = 0; i < sequence->count; i++)
    {
        node_t *node = create_node(&storage[i]);
        add_node(&(sequence->head), node);
    }
}


void free_sequence(node_t *head)
{
    node_t *node;

    for (; head; head = node)
    {
        node = head->next;
        free(head);
    }
}


int is_palindrome(sequence_t *sequence)
{
    int repeats = sequence->count / 2;
    char *start, *end;

    while (repeats > 0)
    {
        start = pop_front(&sequence->head);
        end = pop_back(&sequence->head);

        if (compare_int(start, end) != EQUAL)
            return NOT_PALINDROME;

        repeats--;
    }

    return PALINDROME;
}


node_t* process_sequence(const char *filename, sequence_t *sequence,
int *storage, int (*comparator)(const void *, const void *))
{
    int code = is_palindrome(sequence);

    sequence_t add_sequence;
    init_sequence(&add_sequence);
    add_sequence.count = sequence->count;

    free_sequence(sequence->head);

    create_sequence(&add_sequence, storage);
    node_t *result;

    if (code == PALINDROME)
        result = sort(add_sequence.head, comparator);
    else
        result = reverse(add_sequence.head);

    return result;
}


int write_sequence(const char *filename, sequence_t *sequence)
{
    FILE *out_file = fopen(filename, "w");

    if (!out_file)
    {
        LOG_INFO("%s", "File open error");
        return ERROR_FILE_OPEN;
    }

    int size = sequence->count;

    while (size > 0)
    {
        int *number = (int *)pop_front(&sequence->head);
        size--;
        fprintf(out_file, "%d\n", *number);
    }

    if (fclose(out_file) != OK)
    {
        free_sequence(sequence->head);
        return ERROR_FILE_CLOSE;
    }

    return OK;
}

void print_sequence(sequence_t *sequence)
{
    int size = sequence->count;

    while (size > 0)
    {
        int *number = (int *)pop_front(&sequence->head);
        size--;
        printf("%d ", *number);
    }
}
\end{lstlisting}
\end{center}

\clearpage

\begin{center}
    \captionsetup{justification=raggedright,singlelinecheck=off}
    \begin{lstlisting}[label=lst:main,caption=main.c]
#include <stdio.h>
#include <stdlib.h>
#include "../inc/node.h"
#include "../inc/sequence.h"
#include "../inc/list.h"

int main(int argc, char **argv)
{
    if (argc != COUNT_ARGC)
        return ERROR_ARGC;

    sequence_t sequence, add_sequence;
    int *storage = NULL;

    int code = read_sequence(argv[1], &sequence, &storage);

    if (code)
        return code;

    init_sequence(&add_sequence);
    add_sequence.count = sequence.count;
    create_sequence(&add_sequence, storage);

    print_sequence(&sequence);

    if (is_palindrome(&add_sequence))
        printf("is palindrome\n");
    else
        printf("is not palindrome\n");

    free(storage);

    return OK;
}
\end{lstlisting}
\end{center}

\clearpage

\begin{center}
    \captionsetup{justification=raggedright,singlelinecheck=off}
    \begin{lstlisting}[label=lst:list,caption=list.h]
#ifndef LIST_H
#define LIST_H

#include <stdio.h>
#include <stdlib.h>

#include "node.h"

node_t *reverse(node_t *head);

void sorted_insert(node_t **head, node_t *element,
int (*comparator)(const void *, const void *));

node_t *sort(node_t *head, int (*comparator)(const void *, const void *));

int compare_int(const void *first, const void *second);

#endif
\end{lstlisting}
\end{center}

\begin{center}
    \captionsetup{justification=raggedright,singlelinecheck=off}
    \begin{lstlisting}[label=lst:nodet,caption=node\_t.h]
#ifndef NODE_T_H
#define NODE_T_H

typedef struct node node_t;

struct node
{
    void *data;
    node_t *next;
};

#endif
\end{lstlisting}
\end{center}

\clearpage

\begin{center}
    \captionsetup{justification=raggedright,singlelinecheck=off}
    \begin{lstlisting}[label=lst:node,caption=node.h]
#ifndef NODE_H
#define NODE_H

#include <stdio.h>
#include <stdlib.h>

#include "node_t.h"
#include "macrologger.h"

node_t *create_node(void *data);

void add_node(node_t **head, node_t *node);

void *pop_front(node_t **head);

void *pop_back(node_t **head);

#endif
\end{lstlisting}
\end{center}

\begin{center}
    \captionsetup{justification=raggedright,singlelinecheck=off}
    \begin{lstlisting}[label=lst:sequencet,caption=sequence\_t.h]
#ifndef SEQUENCE_T_H
#define SEQUENCE_T_H

#include "../inc/node_t.h"

typedef struct sequence_t
{
    int count;
    node_t *head;
} sequence_t;

#endif
\end{lstlisting}
\end{center}

\clearpage

\begin{center}
    \captionsetup{justification=raggedright,singlelinecheck=off}
    \begin{lstlisting}[label=lst:sequence,caption=sequence.h]
#ifndef SEQUENCE_H
#define SEQUENCE_H

#include <stdio.h>
#include <stdlib.h>

#include "list.h"
#include "return_codes.h"
#include "constants.h"
#include "macrologger.h"
#include "sequence_t.h"
#include "node_t.h"
#include "node.h"

void init_sequence(sequence_t *sequence);

int read_sequence(const char *filename, sequence_t *sequence,
int **storage);

void create_sequence(sequence_t *sequence, int *storage);

void free_sequence(node_t *head);

int is_palindrome(sequence_t *sequence);

node_t *process_sequence(const char *filename, sequence_t *sequence,
int *storage, int (*comparator)(const void *, const void *));

int write_sequence(const char *filename, sequence_t *sequence);

void print_sequence(sequence_t *sequence);

#endif
\end{lstlisting}
\end{center}

\clearpage

\begin{center}
    \captionsetup{justification=raggedright,singlelinecheck=off}
    \begin{lstlisting}[label=lst:constants,caption=constants.h]
#ifndef CONSTANTS_H
#define CONSTANTS_H

#define COUNT_ARGC     2

#define PALINDROME     1
#define NOT_PALINDROME 0
#define EQUAL          0

#endif
\end{lstlisting}
\end{center}

\begin{center}
    \captionsetup{justification=raggedright,singlelinecheck=off}
    \begin{lstlisting}[label=lst:returncodes, caption=return\_codes.h]
#ifndef RETURN_CODES_H
#define RETURN_CODES_H

#define OK               0
#define ERROR_ARGC       1
#define ERROR_FILE_OPEN  2
#define ERROR_DATA_TYPE  3
#define ERROR_FILE_CLOSE 4
#define ALLOCATE_ERROR   5
#define EMPTY_FILE       6

#endif
\end{lstlisting}
\end{center}