\chapter{Необходимые структуры}

Версия ядра: 5.13.0

\begin{center}
    \captionsetup{justification=raggedright,singlelinecheck=off}
    \begin{lstlisting}
struct filename {
	const char		*name;	/* pointer to actual string */
	const __user char	*uptr;	/* original userland pointer */
	int			refcnt;
	struct audit_names	*aname;
	const char		iname[];
};

struct open_flags {
	int open_flag;
	umode_t mode;
	int acc_mode;
	int intent;
	int lookup_flags;
};

struct audit_names {
	struct list_head	list;		/* audit_context->names_list */

	struct filename		*name;
	int			name_len;	/* number of chars to log */
	bool			hidden;		/* don't log this record */

	unsigned long		ino;
	dev_t			dev;
	umode_t			mode;
	kuid_t			uid;
	kgid_t			gid;
	dev_t			rdev;
	u32			osid;
	struct audit_cap_data	fcap;
	unsigned int		fcap_ver;
	unsigned char		type;		/* record type */
	/*
	 * This was an allocated audit_names and not from the array of
	 * names allocated in the task audit context.  Thus this name
	 * should be freed on syscall exit.
	 */
	bool			should_free;
};

#define EMBEDDED_LEVELS 2
struct nameidata {
	struct path	path;
	struct qstr	last;
	struct path	root;
	struct inode	*inode; /* path.dentry.d_inode */
	unsigned int	flags;
	unsigned	seq, m_seq, r_seq;
	int		last_type;
	unsigned	depth;
	int		total_link_count;
	struct saved {
		struct path link;
		struct delayed_call done;
		const char *name;
		unsigned seq;
	} *stack, internal[EMBEDDED_LEVELS];
	struct filename	*name;
	struct nameidata *saved;
	unsigned	root_seq;
	int		dfd;
	kuid_t		dir_uid;
	umode_t		dir_mode;
} __randomize_layout;

struct open_how {
	__u64 flags;
	__u64 mode;
	__u64 resolve;
};

struct path {
	struct vfsmount *mnt;
	struct dentry *dentry;
} __randomize_layout;
\end{lstlisting}
\end{center}

\chapter{Флаги системного вызова open()}

\texttt{O\_CREAT} --- если файл не существует, то он будет создан;

\texttt{O\_EXCL} --- если используется совместно с \texttt{O\_CREAT}, то при наличии уже созданного файла вызов завершится ошибкой;

\texttt{O\_NOCTTY} --- если файл указывает на терминальное устройство, то оно не станет терминалом управления процесса, даже при его отсутствии;

\texttt{O\_TRUNC} --- если файл уже существует, он является обычным файлом и заданный режим позволяет записывать в этот файл, то его длина будет урезана до нуля;

\texttt{O\_APPEND} --- файл открывается в режиме добавления, перед каждой операцией записи файловый указатель будет устанавливаться в конец файла;

\texttt{O\_NONBLOCK}, \texttt{O\_NDELAY} --- файл открывается, по возможности, в режиме \texttt{non-blocking}, то есть никакие последующие операции над дескриптором файла не заставляют в дальнейшем вызывающий процесс ждать;

\texttt{O\_SYNC} --- файл открывается в режиме синхронного ввода-вывода, то есть все операции записи для соответствующего дескриптора файла блокируют вызывающий процесс до тех пор, пока данные не будут физически записаны;

\texttt{O\_NOFOLLOW} --- если файл является символической ссылкой, то \texttt{open} вернёт ошибку;

\texttt{O\_DIRECTORY} --- если файл не является каталогом, то \texttt{open} вернёт ошибку;

\texttt{O\_LARGEFILE} --- позволяет открывать файлы, размер которых не может быть представлен типом \texttt{off\_t (long)};

\texttt{O\_DSYNC} --- операции записи в файл будут завершены в соответствии с требованиями целостности данных синхронизированного завершения ввода-вывода;

\texttt{O\_NOATIME} --- запрет на обновление времени последнего доступа к файлу при его чтении;

\texttt{O\_TMPFILE} --- при наличии данного флага создаётся неименованный временный обычный файл;

\texttt{O\_CLOEXEC} --- включает флаг \texttt{close-on-exec} для нового файлового дескриптора, указание этого флага позволяет программе избегать дополнительных операций \texttt{fcntl F\_SETFD} для установки флага \texttt{FD\_CLOEXEC}.